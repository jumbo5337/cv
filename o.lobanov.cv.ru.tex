\documentclass[10pt,a4paper]{moderncv}

% moderncv themes
\moderncvtheme[blue]{classic}
\usepackage[utf8]{inputenc}
\usepackage[russian]{babel}                 % replace by the encoding you are using

% adjust the page margins
\usepackage[scale=0.88]{geometry}
%\setlength{\hintscolumnwidth}{3cm}						% if you want to change the width of the column with the dates
%\AtBeginDocument{\setlength{\maketitlenamewidth}{6cm}}  % only for the classic theme, if you want to change the width of your name placeholder (to leave more space for your address details
\AtBeginDocument{\recomputelengths}
% personal data
\firstname{Олег}
\familyname{Лобанов}
\title{Senior Software Developer}
\photo[64pt]{1645901191314.jpeg}
\email{jumbo5337@gmail.com}
\phone{+7 911 197 10 08}
\address{Санкт-Петербург}
%----------------------------------------------------------------------------------
%            content
%----------------------------------------------------------------------------------
\begin{document}
    \maketitle
    \vspace{-10mm}
    % Skills
 \section{Навыки}
    \cvline{\small Languages}{Java, Kotlin, Python, SQL, Bash}
    \cvline{\small Libraries \& Frameworks}{\small Spring Framework (Boot, webflux, security, data), Netty \newline kotlinx.coroutines, java.util.concurrent, JDBC, JAsync }
    \cvline{\small Databases}{Cassandra, CockroachDB, PostgreSQL, YDB}
    \cvline{\small DevOps,Cloud} {GitlabCI, Docker, Kubernetes, Helm}
    \cvline{\small Test} {JUnit, Testcontainers}
    \cvline{\small Other}{Kafka, Gradle, Bazel, REST, Microservices, Distributed Systems, gRPC}
    \section{Иностранные языки}
    \cvlanguage{English}{B2}{}
    % Expirience
    \section{Опыт работы}
    % CRPT
    \cventry{Апрель 2020 - наст.время}{Младший программист-разработчик $\rightarrow$ Старший программист-разработчик}{Честный Знак}{Санкт-Петербург}{}
    {Разрабатываю  Ядро Национальной Системы Маркировки "Честный Знак". \\
    Система ежедневно эмитирует до 1 млрд кодов и обеспечивает их проверку со скоростью до 20к RPS}
    \smallskip
    % Current responsibilites
    {\cvlistitem {Разрабатываю высоконагруженную систему, распределенную через всю Россию}}
    {\cvlistitem {Принимаю участие в обсуждении и планировании Архитектуры системы}}
    {\cvlistitem {Обсуждаю и формализую требования с заказчиками }}
    {\cvlistitem {Инициирую и провожу крупномасштабные рефакторинги}}
    {\cvlistitem {Исследую и внедряю новые технологии}}
    {\cvlistitem {Провожу доклады об опыте внедрения и эксплуатации}}

    {\cvitem{\tiny{c 08/2023 }} {\textit{Старший программист-разработчик}}}
    {\cvlistitem {Разрабатывал географически распределенную (от Москвы до Владивостока) подсистему проверки кодов с SLA 300 ms }
    {\cvlistitem {Увеличил суточную пропускную способность эмиссии с 500 млн до 1 млрд кодов }
    {\cvlistitem {Ускорил подсистему биллинга на 100\% }

    \cvline{\tiny{10/2021 - 08/2023}}{\textit{Программист-разработчик}}
    {\cvlistitem {Внедрил генеративный CD/CD для монорепозитория}
    {\cvlistitem {Установил HPA для сервисов с различными принципами работы}

    \cvline{\tiny{04/2020 - 10/2021}}{\textit{Младший программист-разработчик}}
    {\cvlistitem {Принимал участие в разработке и вводе в эксплуатацию новых подсистем эмиссии и биллинга}}
    {\cvlistitem {Провёл интеграцию с различными  иностранными, государственными, учётными и криптографическими системами}}
    {\cvlistitem {Написал патч для Netty драйвера для лучшей балансировки CockroachDB }}

    % Sleki
    \cventry{\small Октябрь 2019 - Март 2020}{Java Junior Developer}{Sleki}{Санкт-Петербург}{}
    {Разрабатывал микросервисы для приложения по организации досуга}
    % Education
    \section{Образование}
    % SUT
    \cventry{2019--2021}
    { Информационные Системы И Технологии}
    {СПБГУТ}
    {Санкт-Петербург}
    {\textit {Магистратура, с отличием}}
    {}
    % RSHU
    \cventry{2015--2019}{Авиационная Метеорология}{РГГМУ}{Санкт-Петербург}{\textit{Бакалавриат, с отличием}}
    {}
    % SUT Master Thesis
    %\subsection{Thesis's}
    %\cvline{Master}{\emph{Research and development of the intelligent training app}}
    %\cvline{bachelor}{\emph{Influence of surface inhomogenity on the vertical motion at the upper level of atmospheric boundary layer}}
    \end{document}


